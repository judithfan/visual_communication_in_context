\documentclass[9pt,twoside,lineno]{pnas-new}
% Use the lineno option to display guide line numbers if required.

\templatetype{pnassupportinginfo}
% \readytosubmit %% Uncomment this line before submitting, so that the instruction page is removed.

\title{Contextual flexibility in visual communication}
\author{Judith E. Fan, Robert X. D. Hawkins, Mike Wu, and Noah D. Goodman}
\correspondingauthor{Judith Fan\\E-mail: judithfan@gmail.com}

\begin{document}

\maketitle

%% Adds the main heading for the SI text. Comment out this line if you do not have any supporting information text.
\SItext

\subsection*{Subhead}
Type or paste text here. This should be additional explanatory text such as an extended technical description of results, full details of mathematical models, extended lists of acknowledgments, etc.  

\section*{Heading}
\subsection*{Subhead}
Type or paste text here. You may break this section up into subheads as needed (e.g., one section on ``Materials'' and one on ``Methods'').

\subsection*{Materials}
Add a Materials subsection if you need to.

\subsection*{Methods}
Add a Methods subsection if you need to.


%%% Each figure should be on its own page
\begin{figure}
\centering
\includegraphics[width=\textwidth]{example-image}
\caption{First figure}
\end{figure}

\begin{table}\centering
\caption{This is a table}

\begin{tabular}{lrrr}
Species & CBS & CV & G3 \\
\midrule
1. Acetaldehyde & 0.0 & 0.0 & 0.0 \\
2. Vinyl alcohol & 9.1 & 9.6 & 13.5 \\
3. Hydroxyethylidene & 50.8 & 51.2 & 54.0\\
\bottomrule
\end{tabular}
\end{table}

%%% Add this line AFTER all your figures and tables
\FloatBarrier

\bibliography{references}

\end{document}